\documentclass{article}

\usepackage{arxiv}

\usepackage[utf8]{inputenc} % allow utf-8 input
\usepackage[T1]{fontenc}    % use 8-bit T1 fonts
\usepackage{lmodern}        % https://github.com/rstudio/rticles/issues/343
\usepackage{hyperref}       % hyperlinks
\usepackage{url}            % simple URL typesetting
\usepackage{booktabs}       % professional-quality tables
\usepackage{amsfonts}       % blackboard math symbols
\usepackage{nicefrac}       % compact symbols for 1/2, etc.
\usepackage{microtype}      % microtypography
\usepackage{lipsum}
\usepackage{graphicx}

\title{Simultaneously estimating food web complexity and structure with
uncertainty}

\author{
    Anubhav Gupta
    \thanks{Corresponding author}
   \\
    Department of Evolutionary Biology and Environmental Studies \\
    University of Zurich \\
  8057 Zurich, Switzerland \\
  \texttt{\href{mailto:anubhav.gupta@ieu.uzh.ch}{\nolinkurl{anubhav.gupta@ieu.uzh.ch}}} \\
   \And
    Owen L. Petchey
   \\
    Department of Evolutionary Biology and Environmental Studies \\
    University of Zurich \\
  8057 Zurich, Switzerland \\
  \texttt{\href{mailto:owen.petchey@ieu.uzh.ch}{\nolinkurl{owen.petchey@ieu.uzh.ch}}} \\
  }


% Pandoc citation processing

\usepackage{lineno}
\linenumbers
\usepackage {amsmath}
\setlength\parindent{24pt}
\usepackage{setspace}\doublespacing
\usepackage{booktabs}
\usepackage{longtable}
\usepackage{array}
\usepackage{multirow}
\usepackage{wrapfig}
\usepackage{float}
\usepackage{colortbl}
\usepackage{pdflscape}
\usepackage{tabu}
\usepackage{threeparttable}
\usepackage{threeparttablex}
\usepackage[normalem]{ulem}
\usepackage{makecell}


\begin{document}
\maketitle

\def\tightlist{}


\begin{abstract}
\begin{enumerate}
\def\labelenumi{\arabic{enumi})}
\tightlist
\item
  Food web models explain and predict the trophic interactions in a food
  web, and they can infer missing interactions among the organisms. The
  allometric diet breadth model (ADBM) is a food web model based on the
  foraging theory. In the ADBM the foraging parameters are
  allometrically scaled to body sizes of predators and prey. In Petchey
  et al. (2008), the parameterisation of the ADBM had two limitations:
  (a) the model parameters were point estimates, and (b) food web
  connectance was not estimated.
\item
  The novelty of our current approach is: (a) we consider multiple
  predictions from the ADBM by parameterising it with approximate
  Bayesian computation, to estimate parameter distributions and not
  point estimates. (b) Connectance emerges from the parameterisation, by
  measuring model fit using the true skill statistic, which takes into
  account prediction of both the presences and absences of links.
\item
  We fit the ADBM using approximate Bayesian computation to 16 observed
  food webs from a wide variety of ecosystems. Connectance was
  consistently overestimated in the new parameterisation method. In some
  of the food webs, considerable variation in estimated parameter
  distributions occurred, and resulted in considerable variation
  (i.e.~uncertainty) in predicted food web structure.
\item
  We conclude that the observed food web data is likely missing some
  trophic links that do actually occur, and that the ADBM likely
  predicts some links that do not exist. The latter could be addressed
  by accounting in the ADBM for additional traits other than body size.
  Further work could also address the significance of uncertainty in
  parameter estimates for predicted food web responses to environmental
  change.
\end{enumerate}
\end{abstract}

\keywords{
    connectance
   \and
    ABC
   \and
    ADBM
   \and
    food web
   \and
    true skill statistic
   \and
    uncertainty
  }

\hypertarget{introduction}{%
\section{Introduction}\label{introduction}}

Knowledge about the trophic interactions among the organisms in a
community is crucial for understanding the structure and dynamics of
ecological communities and for predicting their response to
environmental change (Dunne, Williams, and Martinez 2002; Tylianakis and
Binzer 2014; O'Connor et al. 2009; Bergamino, Lercari, and Defeo 2011;
Krause et al. 2003; Lurgi, L\a'opez, and Montoya 2012; Morris, Sinclair,
and Burwell 2015). The network of trophic interactions is often referred
to as a food web. The food web structure can provide answers to key
ecological questions: which species are more vulnerable to environmental
changes such as temperature (Petchey et al. 1999); how robust a food web
is to extinctions (Dunne, Williams, and Martinez 2002); and how a food
web reacts if the top predators are removed (Knight et al. 2005)?

Trophic interactions information from multiple sources can be used to
infer a food web, e.g.~gut contents (Peralta-Maraver,
L\a'opez-Rodríguez, and de Figueroa 2017), stable isotope composition of
tissues (Layman et al. 2007), and experimentation (Warren 1989).
Sometimes the methods used to infer the interactions may lead to
uncertainty in the constructed food web. E.g. In gut content analysis of
some fish predators, there can be tissues which are not identifiable and
cannot be assigned with certainty to a specific prey item (Baker,
Buckland, and Sheaves 2014). With stable isotope ratios of tissues,
uncertainty may be due to factors such as variability in the isotopic
fractionation values across multiple combinations of diets and
tissues/species, unquantified temporal or spatial variation in prey
isotopic values, and variation caused by routing of particular dietary
nutrients into particular tissues (Crawford, Mcdonald, and Bearhop
2008). Furthermore, complete recording of all interactions usually
requires a large sampling effort even at small spatial and temporal
scales (Hobson, Piatt, and Pitocchelli 1994). Food web structure is very
difficult to record at larger spatial and temporal scales without losing
resolution (spatial, temporal, and taxonomic) (Gravel et al. 2013;
Martinez 1991; Jord\a'an and Osv\a'ath 2009). Less than complete
sampling of interactions can result in no interaction being observed
between a pair of individuals that in fact do interact, which results in
missing links in a food web. Due to under-sampling, food webs can be
poorly understood, which may hinder further advances in the field
(Martinez et al. 1999).

When interactions are difficult to observe, and hence well-documented
food webs are not available, models which predict species interactions
may provide a solution (Tamaddoni-Nezhad et al. 2013; Gravel et al.
2013; Petchey et al. 2008; Allesina, Alonso, and Pascual 2008; Cohen,
Newman, and Steele 1985). A food web model can be used to predict
missing information about species interactions. For example, Petchey et
al. (2008) showed how a model of species interactions (and therefore
food web structure) could be parameterised from data on the known
presence and absence of trophic interactions. The model and its
parameter values encode the rules for occurrence or absence of species
interactions to predict food web structure. Observed data may be used to
select and parameterise the model. Tamaddoni-Nezhad et al. (2013) used
large agricultural datasets, logic-based machine learning and text
mining to assign interactions between nodes to automatically construct
food webs. Gravel et al. (2013), inspired by the niche model of food web
structure developed a method that used the statistical relationship
between predator and prey body size to infer the food web.

Food web models are also useful for ecological forecasting. Lindegren et
al. (2010) used a stochastic food web model driven by regional climate
scenarios to produce quantitative forecasts of cod dynamics in the
twenty-first century. Hattab et al. (2016) forecasted the potential
impacts of climate change on the local food web structure of the highly
threatened Gulf of Gabes ecosystem, located in the south of the
Mediterranean Sea. Hence, food web models have an important role in
filling gaps in knowledge about species interactions, including
predicting future changes in food web structure.

The allometric diet breadth model (ADBM) was the first model able to
predict food web complexity and structure (Beckerman, Petchey, and
Warren 2006; Petchey et al. 2008). The ADBM uses foraging theory,
specifically the contingency model (MacArthur and Pianka 1966), to
predict which set of the available prey species would be consumed by a
predator. This set is the prey species that maximises the energy intake
rate of the predator. The model requires the foraging related traits of
species, such as energy content of a potential prey item, the rate of
space clearance (also known as attack rate), the density of prey items,
and handling time (the amount of time required to handle food items).
The model is termed ``allometric'' because each of these quantities is
derived from the body size of the prey and predator using several
allometric relationships. The ADBM has also been used to investigate the
effect of temperature on an observed food web structure (O'Gorman et al.
2019).

The ADBM had variable success in explaining the structure of 15
different food webs, with the proportion of links correctly predicted
ranging from 5 \% to 65 \% (Table \ref{fig:tab_1}). The ADBM correctly
predicted 65\% of the presence of links in the Coachella valley food
web. The poorest prediction of presence of links was for the Grasslands
food web with only 7\% of the presence of links correctly predicted.
When trophic interactions were more strongly dependent on size, the
model correctly predicted a greater proportion of links. Indeed,
constructing a food web based only on body size (i.e.~ignoring taxonomy)
resulted in almost twice the number of correctly predicted links,
i.e.~83\%, in contrast to taxonomy (Woodward et al. 2010).

Although Petchey et al. (2008) demonstrated that foraging theory could
predict food web structure, their implementation of the ADBM included at
least two limitations. The parameterisation method provided estimates of
the parameters with no uncertainty: a single set of parameter values
that maximised the explanatory power was selected. In other words, the
parameterisation method led to point estimates of the parameters that
predicted a single food web structure (because the ADBM is purely
deterministic). Moreover, the best predicted food web was not exactly
the same as the observed one. In a sense then, the parameterisation
method used in Petchey et al. (2008) was akin to estimating the
intercept and slope of a regression line, but not any uncertainty in
those parameters. Given that uncertainty is an essential dimension in
ecological models, and in predictions about the future state of
ecological communities (Petchey et al. 2015; Carpenter 2016), this is an
important limitation.

The second limitation was in the estimation of the connectance of the
food web, which is the number of realised trophic links divided by the
total number of possible trophic links. Although the ADBM can in
principle predict connectance, Petchey et al. (2008) prevented the model
from doing so. They set the value of relevant parameters in the model to
instead ensure the predicted connectance was equal to the observed
connectance. The ADBM was not therefore used to simultaneously predict
complexity and structure of food webs. Moreover, fixing predicted
connectance to be equal to observed connectance does not account for the
possibility that the observed connectance was imperfectly measured.
Indeed, if low effort was used to observe the trophic links in a
community, the observed connectance are likely to be lower than if all
trophic links were observed. Connectance is an important driver for the
stability and dynamics of a food web (May 1972) and most of the
structural properties of food webs co-vary with connectance (Dunne,
Williams, and Martinez 2002; Poisot and Gravel 2014), thus this
limitation must be addressed.

In this article we report on how we address these limitations. We
removed the first limitation by applying approximate Bayesian
computation (ABC). The approach originated in population genetics and
has been used in a wide range of research fields: systems biology (Toni
et al. 2009), ecology (Jabot and Chave 2009), epidemiology (Shriner et
al. 2006) and ecological networks (Ibanez 2012; Poisot and Stouffer
2016). One of the advantages of ABC is that it does not require a
likelihood function. As ADBM is a complex deterministic model where the
likelihood can not be explicitly evaluated, ABC is a good choice of
parameterisation.

We addressed the second limitation by allowing estimation of number of
links as well as arrangement of links. To accomplish this, we measured
model fit by using the true skill statistic, which takes into account
both the number of presences and absences of links correctly predicted.
High values of the true skill statistic occurs when both the predicted
arrangement of links and the predicted number of links are close to the
observed arrangement and number of links, respectively.

\newgeometry{margin=1cm}
\begin{landscape}\begin{table}

\caption{\label{tab:unnamed-chunk-1}\label{fig:tab_1}Information about the food webs predicted using the ADBM.}
\centering
\resizebox{\linewidth}{!}{
\fontsize{7}{9}\selectfont
\begin{tabular}[t]{>{\raggedright\arraybackslash}p{3cm}|>{\raggedright\arraybackslash}p{8em}|l|l|l|l|>{\raggedright\arraybackslash}p{8em}|>{\raggedright\arraybackslash}p{8em}|>{\raggedright\arraybackslash}p{8em}}
\hline
Common food web name & Predation matrix source & Body size source & General ecosystem & Number of species & Connectance & Body size range (approximate) & Proportion of presence of links correct & Type of interactions\\
\hline
Benguela Pelagic & (Yodzis 1998) & (Yodzis 1998) & Marine & 29 & 0.23 & $10^{-8}$ to $10^6$ & 0.57 & Predation\\
\hline
Broadstone Stream (taxonomic aggregation) & (Woodward and Hildrew 2001; Woodward
et al. 2005) & (Brose et al. 2005) & Freshwater & 29 & 0.19 & $10^{-6}$ to $10^{-2}$ & 0.62 & Predation\\
\hline
Broom & Memmott et al. 2000) & (Brose et al. 2005) & Terrestrial & 68 & 0.02 & $10^{-6}$ to $10^0$ & 0.08 & Herbivory, Parasitism, Predation, Pathogenic\\
\hline
Capinteria & (Lafferty et al. 2006) & (Reide, unpublished) & Marine (Salt Marsh) & 72 & 0.05 & $10^{-14}$ to $10^5$ & 0.16 & Predator-parasite, Parasite-parasite\\
\hline
Caricaie Lakes & (Cattin et al. 2004) & (Brose et al. 2005) & Freshwater & 158 & 0.05 & $10^{-5}$ to $10^5$ & 0.13 & Predation, Parasitism\\
\hline
Coachella Valley & (Polis 1991) & (Reide, unpublished) & Terrestrial (Desert) & 26 & 0.34 & $10^{-8}$ to $10^4$ & 0.65 & Herbivory, Predation\\
\hline
EcoWEB41 & (Cohen 1989) & (Jonsson 1998) & Marine & 19 & 0.14 & $10^{-11}$ to $10^6$ & 0.47 & Predation\\
\hline
EcoWEB60 & (Cohen 1989) & (Jonsson 1998) & Terrestrial & 33 & 0.06 & $10^{-5}$ to $10^6$ & 0.24 & Predation, Parasitism, Herbivory\\
\hline
Grasslands & (Dawah et al. 1995) & (Brose et al. 2005) & Terrestrial & 65 & 0.03 & $10^{-3}$ to $10^{-2}$ & 0.07 & Herbivory, Parasitism\\
\hline
Mill Stream & (Ledger, Edwards, Woodward unpublished) & (Brose et al. 2005) & Freshwater & 80 & 0.06 & $10^{-6}$ to $10^{-1}$ & 0.37 & Herbivory, Predation\\
\hline
Sierra Lakes & (Harper-Smith et al. 2005) & (Brose et al. 2005) & Freshwater & 33 & 0.16 & $10^{-4}$ to $10^0$ & 0.60 & Predation\\
\hline
Skipwith Pond & (Warren 1989) & (Brose et al. 2005) & Freshwater & 71 & 0.07 & $10^{-4}$ to $10^{-1}$ & 0.14 & Predation\\
\hline
Small Reef & (Opitz 1996) & (Reide unpublished) & Marine (Reef) & 50 & 0.22 & $10^{-11}$ to $10^5$ & 0.41 & Predation, Herbivory\\
\hline
Tuesday Lake & (Jonsson et al. 2005) & (Brose et al. 2005) & Freshwater & 73 & 0.08 & $10^{-11}$ to $10^3$ & 0.46 & Predation\\
\hline
Ythan & (Emmerson and Raffaelli 2004) & (Emmerson and Raffaelli 2004) & Marine (Estuarine) & 88 & 0.05 & $10^{-12}$ to $10^0$ & 0.22 & Predation\\
\hline
Broadstone Stream (size aggregation) & (Woodward
et al. 2010) & (Woodward
et al. 2010) & Freshwater & 29 & 0.24 & $10^{-7}$ to $10^2$ & 0.83 & Predation\\
\hline
\end{tabular}}
\end{table}
\end{landscape}
\restoregeometry

\hypertarget{materials-and-methods}{%
\section{Materials and Methods}\label{materials-and-methods}}

In the upcoming sections, we present a detailed account of the
application of ABC to parameterise the ADBM, the description of the ADBM
and of the food web data we used. We explain the rejection Monte Carlo
ABC method in the main text, and Markov chain Monte Carlo ABC and
sequential Monte Carlo ABC methods in the Supplementary information
(hereafter SI) Section S1 (hereafter SI-S1). We computed an accuracy
measure known as true skill statistic to assess the ADBM's predictions
and calculated different food web properties to compare these
predictions across food webs.

\hypertarget{allometric-diet-breadth-model-adbm}{%
\subsection{Allometric Diet Breadth Model
(ADBM)}\label{allometric-diet-breadth-model-adbm}}

The allometric diet breadth model (ADBM) is based on optimal foraging
theory, specifically the contingency foraging model (MacArthur and
Pianka 1966). The ADBM predicts the set of prey species a consumer
should feed upon to maximise its rate of energy intake (Petchey et al.
2008) (hereafter referred as PBRW study). The species in this set are
assumed to have the trophic link with the predator. To make these
predictions, the model assumes that a foraging predator is in one of two
exclusive states: searching for prey or handling a prey item. The model
requires four variables for each potential predator-prey interaction:

\begin{itemize}
\tightlist
\item
  The energy content of the resources \(E_i\) (only prey \(i\) specific)
  (energy).
\item
  The handling times \(H_{ij}\), which is the time not spent searching
  caused by consuming a prey item (prey \(i\) and predator \(j\)
  specific) (time).
\item
  The space clearance rates \(A_{ij}\) (also known as the attack rate;
  prey \(i\) and predator \(j\) specific) (area or volume per time).
\item
  The prey densities \(N_i\) (only prey \(i\) specific) (individuals per
  area or volume).
\end{itemize}

The term ``Allometric'' in the ADBM refers to the use of four allometric
relationships, one for each of these four variables, including the body
size of the predator \(M_j\) and prey \(M_i\) (Table \ref{fig:tab_2}).
With these four allometric relationships, and the body size of each of
the species in a community, we can predict the four variables (energy,
handling time, space clearance rate, and prey density), and then use the
contingency foraging model to predict diets.

Each of the four allometric equations has parameters: a constant and/or
at least one exponent (Table \ref{fig:tab_2}). It is the value of some
of these parameters that can be estimated to have the predicted food web
structure match (as closely as possible) the structure of an observed
food web. This is akin to choosing values of slope and intercept of a
linear regression that maximises the fit of the regression line to the
observed data.

Because some of the allometric constants and exponents are known, and
because others are redundant with respect to each other (see Table
\ref{fig:tab_2} for details), we estimate only the following parameters:
\(a\), \(a_i\), \(a_j\) and \(b\) in the model (Table \ref{fig:tab_2}).

\begin{table}

\caption{\label{tab:unnamed-chunk-2}\label{fig:tab_2}Traits with their allometric function and corresponding parameters in ADBM.}
\centering
\resizebox{\linewidth}{!}{
\begin{tabular}[t]{l|l|l|l}
\hline
Traits (Unit) & Allometric function & Parameters & Comments\\
\hline
Energy $(Joules)$ & $E_i = eM_i$ & $e$ & Arbitrary. No effect on structure\\
\hline
Abundance $(\text{individual}/m^2 \text{ or } \text{individual}/m^3)$ & $N_i = nM_i^{n_i}$ & $n$ & Connectance affected by the product $nah^*$\\
\hline
 &  & $n_i$ & Assumed value of $\frac{-3}{4}$ based on empirical data\\
\hline
Space Clearance Rate $(m^2/s \text{ or }  m^3/s)$ & $A_{ij} = aM_i^{a_i}M_j^{a_j}$ & $a$ & Connectance affected by the product $nah^*$;\\
\hline
 &  &  & Estimated using ABC\\
\hline
 &  & $a_i$ & Estimated using ABC\\
\hline
 &  & $a_j$ & Estimated using ABC\\
\hline
Handling time $(s)$ & $H_{ij} = \frac{h}{b-\frac{M_i}{M_j}} \text{ if } \frac{M_i}{M_j} < b$ & $h$ & Connectance affected by the product $nah^*$\\
\hline
 & $H_{ij} = \infty \text{ if } \frac{M_i}{M_j} \geq b$ & $b$ & Estimated using ABC\\
\hline
\end{tabular}}
\end{table}

\hypertarget{observed-food-web-data}{%
\subsection{Observed food web data}\label{observed-food-web-data}}

The observed food webs that we fit the ADBM to belong to marine,
freshwater and terrestrial ecosystems (Table \ref{fig:tab_1}). The
observed connectance of these food webs is from 0.02 to 0.34 and there
are 19 to 158 species. The food webs contain primary producers,
herbivores, carnivores, parasites, and parasitoids. They also contain
various types of feeding interactions, including predation, herbivory,
bacterivory, parasitism, pathogenic, and parasitoid.

The goodness of fit of the ADBM's predictions depends on the types of
interactions in the food webs in the PBRW study. Because some of the
interactions are more size structured than other interactions.
Predacious and aquatic herbivore interactions were predicted better than
parasitoid and herbivory ones (PBRW study). Moreover, the ADBM's
predictions improve when the species (taxonomic) are replaced by size
classes (ignoring taxonomy) (Woodward et al. 2010).

All food webs with one exception (Broadstone Stream) was available only
at the species level, i.e.~with information about interactions between
species and the body size of species. We use the term ``species'' in
this study to indicate a ``node'' in a food web in which nodes are
connected by trophic interactions, and nodes are a collection of
individuals that share links. These species/nodes are not always
taxonomic species, but can be broader taxonomic ranks.

In contrast, the Broadstone Stream food web data contained interactions
between individuals and the individual body sizes. Thus, the Broadstone
Stream food web can be constructed by aggregating by either taxonomy or
size (Woodward et al. 2010). The ADBM can predict the food web
irrespective of the aggregation method, and aggregation by size,
ignoring taxonomy led to higher explanatory power i.e.~the match between
observed and predicted food web structure was higher. With aggregation
by size, 83\% of the links were correctly predicted than with taxonomic
aggregation where 40\% of links were correctly predicted (Woodward et
al. 2010).

\hypertarget{parameter-estimation-approximate-bayesian-computation}{%
\subsection{Parameter estimation: Approximate Bayesian
Computation}\label{parameter-estimation-approximate-bayesian-computation}}

We used approximate Bayesian computation (ABC) to identify sets of
parameter values that resulted in predicted food webs that were close in
structure to the observed food web. ABC is an approach that does not
require a likelihood function. Instead, there is a distance function
that measures the distance between a model's prediction and the observed
data. The approximation of the likelihood depends on the ABC method
used, which is further discussed below and SI. The model parameter
values are sampled from a prior distribution. The accepted parameter
values form an approximate posterior distribution for the model
parameter. We implemented three ABC methods to parameterise the ADBM:
namely rejection Monte Carlo (Fig. \ref{fig:fig_m1}), Markov chain Monte
Carlo, and sequential Monte Carlo. The three methods produced very
similar results (SI Figs S33-S34) and we therefore only include the
simplest (rejection) in this main text.

\begin{figure}

{\centering \includegraphics[width=400px]{../fig/schematic} 

}

\caption{\label{fig:fig_m1} Flowchart of rejection approximate Bayesian computation method implemented to parameterise the ADBM.}\label{fig:unnamed-chunk-3}
\end{figure}

\hypertarget{prior-distribution}{%
\subsubsection{Prior distribution}\label{prior-distribution}}

The prior distributions for \(a_i\) and \(a_j\) were chosen to be
uniform distributions. The range of distribution was from -1.5 to 1.5
and 0 to 3 for \(a_i\) and \(a_j\) respectively, informed by the
estimates in Rall et al. (2012). However, we chose a prior range
specific to food webs for the parameter \(b\) because body size varies
greatly among the species in the observed food webs. For example: in the
Benguela Pelagic food web, the body sizes of species range from the
order of \(10^{-8}\) gm to \(10^5\) gm. Hence, the range of
prey-predator ratio was from the order of \(10^{-14}\) to \(10^{14}\).
To take this into account, we took the prior of \(log_{10}(b)\) from a
uniform distribution ranging from \(-15\) to \(15\). In the case of
parameter \(a\), we chose the prior of \(log_{10}(a)\) to be a uniform
distribution. However, the prior range varied between food webs. For
example, the prior range for Benguela Pelagic was chosen to be -12 to
10. The method for choosing the specific range of the prior distribution
of the \(a\) parameter is detailed in SI-S4.

\hypertarget{comparison-of-observed-and-predicted}{%
\subsubsection{Comparison of observed and
predicted}\label{comparison-of-observed-and-predicted}}

The difference between the model's prediction and the observed data
(e.g.~the sum of squared residuals is such a distance in linear
regression) is quantified by a distance measure. The distance is lower
when there is a closer match between the model's prediction and the
observation. A perfect match would result in zero distance.

The magnitude of the distance is used for the acceptance or rejection of
a set of parameter values. An accepted set of parameter values
contributes to the posterior distribution, rejected ones do not. This
makes the distance measure one of the important features of ABC. A
threshold distance is chosen, and if the distance for a particular set
of parameter values is less than the threshold, then that set of
parameter values contributes to the posterior distribution. When the
distance is greater than the threshold, the parameter values do not
contribute to the posterior. Hence, the magnitude of the distance
threshold determines the proportion of a model's parameters that are
accepted. A higher threshold causes a high proportion of acceptances but
less accuracy with the acceptance of some parameter sets that result in
predictions quite unlike the observed data. Below, we first describe and
justify our choice of distance measure, and then our choice of
threshold.

In the PBRW study the measure of distance was equivalent to
\(1 - a / (a + c)\), where \(a\) is the number of observed links that
were predicted (the number of true positives) and \(c\) is the number of
observed links that were not predicted (the number of false negatives).
A distance of 0 indicates that all observed links were correctly
predicted. One way for the ADBM to achieve this is to predict that every
species has a trophic link with every other species including itself --
a fully connected food web with connectance of 1. The PBRW study
prevented this by constraining the number of predicted links to be equal
to the number of observed links, i.e.~the model connectance was fixed to
be the same as the observed connectance. In this study, we relaxed this
constraint, with the number of links as well as the arrangement of links
being estimated. The first step was to choose an appropriate distance
measure.

The distance measure used in this study is 1 minus the true skill
statistic: \(\text{distance} = 1 - \text{TSS}\). This distance ranges
from 0 to 2.

TSS is defined as:

\[ \text{TSS} = \frac{ad-bc}{(a+c)(b+d)} \] where \(a\) is the number of
observed links that are predicted by the model (true positives), \(d\)
is the number of observed absences of links that are correctly predicted
(true negatives), \(b\) is the number of false positives, and \(c\) is
the number of false negatives.

The \(TSS\) ranges from \(-1\) to \(1\), where +1 indicates a perfect
prediction. A \(TSS\) value of zero or less indicates a performance no
better than random.

The inclusion of true and false negatives in the distance measure means
that the best theoretically possible prediction (smallest distance) is a
unique prediction, and specifically the one in which the predicted
presence and absence of links matches exactly with the observed presence
and absence of links.

Food web dynamics and stability are strongly dependent on connectance
(May 1972), we therefore set the distance threshold (for acceptance)
such that the model had a reasonable chance of predicting the observed
value of connectance.

To do this, we examined how the predicted connectance varied with the
distance threshold. An example of this relationship is given in Fig.
\ref{fig:fig_m2} for the Benguela Pelagic food web. We chose the minimum
threshold value that gave a range of predicted connectance containing
the observed connectance.

Furthermore, it is useful to note that in Fig. \ref{fig:fig_m2} there
are no connectance values below a distance threshold value of less than
0.5 because for this particular food web there were no sets of parameter
values that achieved a better model fit than is indicated by
\(1-TSS = 0.5\). I.e. it is impossible for the ADBM to make better
predictions than this. One reason for this is that the ADBM, when body
size is the only trait, can only predict contiguous diets in trait
space, whereas the observed data contains gaps in the diet.

\begin{figure}[h]

{\centering \includegraphics[width=200px]{../fig/Benguela_Pelagic_connectanceCI_vs_tol} 

}

\caption{\label{fig:fig_m2}The prediction interval of the predicted connectance increases with increasing distance threshold for the Benguela Pelagic food web. The green line and black line represent the observed connectance and mean of predicted connectance respectively.}\label{fig:unnamed-chunk-4}
\end{figure}

\hypertarget{the-rejection-abc-method}{%
\subsubsection{The Rejection ABC
method}\label{the-rejection-abc-method}}

In the rejection ABC method, a set of parameter values are sampled from
the prior distributions. This set of parameter values is either
accepted, and thereby added to the posterior distribution of the
parameter values, or it is rejected (based on if the distance 1 -
\(TSS\) is less than or greater than the threshold distance, as
mentioned above). This process is repeated until there are enough
acceptances to give stable (approximate) posterior distributions. In
addition, we used a kernel function that assigns weight to each set of
parameter values, where the weight is inversely proportional to the
distance (1 - \(TSS\)).

In the upcoming section, we further detail the rejection ABC method.

\emph{Properties:}

\begin{itemize}
\item
  A prior distribution \(\pi(\theta)\): \(\pi\) is the uniform
  distribution for parameters \(\theta = (a, a_i, a_j, b)\)
\item
  A model prediction \(model(\theta)\): ADBM\((\theta)\). This is a
  predicted food web, \(x_i\), given by a particular set of parameter
  values \(\theta_i\). Hence, \(x_i = ADBM(\theta_i)\)
\item
  A summary statistic \(s(x)\): \(x\) is the predation matrix predicted
  by the ADBM.
\item
  \(\begin{aligned} \text{A kernel function } K(u): \text{ epanechnikov } K(u) &= \frac{3}{4}(1-\frac{u}{tol}) \text{ if } u \leq tol \\ &= 0 \text{ otherwise} \end{aligned}\)
\end{itemize}

where \(tol\) is the distance threshold

\begin{itemize}
\item
  A distance function \(d(x_i,y)\): \(d(x_i,y) = 1 - TSS(x_i,y)\)
\item
  An observed food web \(y\), in the form of a predation matrix
  containing zeros and ones.
\end{itemize}

\emph{Sampling:}

for \(i = 1 \dots n = 1000\)

\begin{itemize}
\item
  Draw a set of parameter values \(\theta_i\) from the prior
  distribution \(\pi(\theta)\).
\item
  Compute the model result \(x_i = model(\theta_i)\)
\item
  Compute \(s(x_i)\) and \(d(s(x_i), s(y))\)
\item
  Accept or reject the parameter set probabilistically:

  \begin{itemize}
  \item
    Assign a probability \(p_i\) to \(\theta_i\) as per the kernel
    \(K\); \(p = K(d)\), where \(d\) is the distance evaluated in the
    previous step.
  \item
    Compute \(\alpha \sim U(0,1)\)
  \item
    If \(p_i \leq \alpha\), then accept \(\theta_i\) and \(i = i + 1\)
  \end{itemize}
\end{itemize}

\emph{Output:}

An approximate joint posterior distribution using the accepted
\(\theta_1, \dots, \theta_n\).

\hypertarget{assessment-of-model-fit}{%
\subsection{Assessment of model fit}\label{assessment-of-model-fit}}

Accuracy is how close the model prediction is to the observation. The
ADBM's prediction is a predation matrix that consists of the presence
and absence of links thus comparing how close the prediction is to the
observation is not straightforward as comparing two numerical values. We
defined the accuracy of the ADBM using true skill statistics to take
into account the true and false predictions of both the presence and
absence of links, which is defined above.

We examined how closely structural properties of the predicted food web
matched those of the observed food webs. We evaluated properties such as
proportion of basal species, proportion of intermediate species,
proportion of top species, proportion of herbivores, mean omnivory,
clustering coefficient, standard deviation of generality, standard
deviation of vulnerability, diet similarity, mean path length and
nestedness. We did not compute mean trophic level and maximum trophic
level because their computation did not converge in the R \emph{cheddar}
package (Hudson et al. 2013) for all the food webs.

We investigated the performance of the ADBM parameterised with the ABC
by computing standardised error of the food web properties, where the
standardised error is the absolute raw error (the difference between
observed and predicted value) divided by the maximum absolute raw error
for that property. We did not calculate the standardised error for mean
omnivory and mean path length because it had some NA values and infinite
values for all the food webs respectively.

\hypertarget{results}{%
\section{Results}\label{results}}

As an example of the model outcomes, we first present the results for
the Benguela food web (e.g.~predicted food web structure, variation in
predicted food web structure, and posterior parameter distributions). We
chose this food web as it was well explained using the method of Petchey
et al (2008) (hereafter referred as PBRW study). The results of the
other food webs are included in the SI Figs S1-S32. We then compare
model outcomes across all empirical food webs between the PBRW study and
our current work. We compare the true skill statistic of the two
approaches and compare some food web properties, such as proportions of
basal, intermediate, and top species.

The true skill statistic (TSS) of the predicted Benguela Pelagic food
web varied between 0.4 and 0.52. This variation in the TSS is
represented in terms of predation matrices displayed in Fig.
\ref{fig:fig_r1}(a), which overlays 1000 independent predation matrices
created from the posterior parameter distributions. In all the 1000
independent predation matrices, the predicted links are mostly present
in the upper triangular portion of the matrix where most of the observed
links are also present. Links in the upper right triangle of the
predation matrix are for predators feeding on prey smaller than
themselves.

In the 1000 predicted predation matrices, there predators are sometimes
smaller than their predicted prey, the links in the lower left triangle
of the predation matrix. This is also portrayed in the marginal
distribution of \(log_{10}(b)\) in Fig. \ref{fig:fig_r3}(d), as it
includes values greater than \(b=2\) (\(log_{10}(b)=0.3\)). This is
relevant as values of \(b=2\) make the most profitable prey item equal
in size to the predator size. Lower values of \(b\) make the most
profitable prey item smaller than the size of the predator.

There were around 250 potential links in the lower left triangle of the
predation matrix that were never predicted in any of the 1000 predicted
predation matrix (Fig. \ref{fig:fig_r1}(b)). This strongly suggests that
the predator-prey size ratio of these links is so small (i.e.~very large
prey, very small predator) that the links cannot occur, given that the
preponderance of observed links are predators consuming prey smaller
than themselves.

\begin{figure}

{\centering \includegraphics[width=300px]{../fig/Benguela_Pelagic_pred_mat} 

}

\caption{\label{fig:fig_r1} (a) Observed and predicted predation matrices for Benguela Pelagic food web. Body size increases from left to right and top to bottom along the predation matrix. Black circles show where there is an observed trophic link. The intensity of the pink circles shows the proportion of 1000 predicted food webs that had a trophic link between the corresponding species. This type of overlay is shown for two examples predicted in panel (c). (b) The histogram of the number of times a link was predicted across 1000 independently predicted food webs. There were 841 species pairs in this food web. About 150 of these were predicted to have a trophic link in all 1000 predicted predation matrices. The red bar shows the number of pairs of species for which a trophic link was never predicted. (c) Two predicted predation matrices for Benguela Pelagic food web corresponding to the minimum and the maximum value of estimated $b$, and their sum.}\label{fig:unnamed-chunk-5}
\end{figure}

The marginal posterior of parameter \(b\) in the Benguela Pelagic food
web was more constrained than the marginal posterior distribution of the
other three allometric parameters (Fig. \ref{fig:fig_r3}) as the
posterior range was the narrowest.

\begin{figure}

{\centering \includegraphics[width=300px]{../fig/Benguela_Pelagic_distribution} 

}

\caption{\label{fig:fig_r3} Marginal prior and marginal posterior distribution of the ADBM parameters for the Benguela Pelagic food web estimated using rejection ABC.}\label{fig:unnamed-chunk-6}
\end{figure}

\begin{figure}

{\centering \includegraphics[width=500px]{../fig/ABC_vs_point_estimates} 

}

\caption{\label{fig:fig_r2} TSS (a), connectance (b) and ADBM parameters (c, d, e, f) computed using the ABC method compared with the corresponding point estimates from Petchey et al (2008). The red lines are the 95\% credible/prediction intervals and the black filled circles represent the corresponding means. The grey region represents the intervals of the prior distributions for $a_i$ and $a_j$. The grey lines represent the prior range of the parameters $a$ and $b$ in the $log_{10}$ scale. The prior range for the parameter $b$ extends above and below the y-axis limits for some food webs and so the values of the limits are shown on the plot. The dashed black lines are the 1:1 relationships for reference.}\label{fig:unnamed-chunk-7}
\end{figure}

The mean true skill statistic using the ABC approach was higher than the
point estimates from the PBRW study (Fig. \ref{fig:fig_r2}(a)) across
all food webs except one. Our present approach led to estimates of
connectance greater than the values of connectance of the PBRW study,
which were fixed to equal the observed values of connectance.

We did not find a consistent relationship between the parameters
estimated using the current approach and those estimated in the PBRW
study (Fig. \ref{fig:fig_r2}(c-f)), except for in the case of parameter
\(b\). The mean using the ABC approach was always higher than the
estimates from the PBRW study (Fig. \ref{fig:fig_r2}(f)) and the 95\%
credible interval of the posterior of \(b\) includes the estimate from
the PBRW study.

The marginal posterior of parameter \(b\) was more constrained than the
other three allometric parameters, i.e.~the posterior range was the
narrowest (SI Figs S17-S32). In most of the food webs, the parameter
\(b\) had a unimodal distribution (SI Figs S17-S32). EcoWEB60 and
Grasslands had a bimodal distribution and Sierra Lakes had three modes.

The structural food web properties proportion of intermediate species,
mean omnivory, clustering coefficient, sd of generality, sd of
vulnerability, diet similarity and nestedness estimated from the current
ABC approach were generally higher than the PBRW study (SI Fig. S36(b,
e-j)). The properties proportion of basal species, proportion of top
species, and proportion of herbivores were generally lower (SI Fig.
S36(a, c, d)).

The real values of the proportion of intermediate species, mean
omnivory, clustering coefficient, sd of generality, sd of vulnerability
and nestedness, were mostly within the lower range of the predicted 95\%
interval. The proportion of basal species, proportion of top species,
proportion of herbivores were underestimated in comparison to the real
values for most of the food webs.

The ADBM, when parameterised with the ABC, generally better predicted
the structural food web properties, such as proportion of basal species
when the true skill statistics was higher (Fig. \ref{fig:fig_r5}(a))
across the 16 food webs. However, the ABC parameterised ADBM less
accurately predicted food web properties on average than in the PBRW
study (Fig. \ref{fig:fig_r5}(b)).

\begin{figure}

{\centering \includegraphics[width=300px]{../../manuscript/fig/fig7} 

}

\caption{\label{fig:fig_r5} (a) The mean standardised error of the food web properties predicted from the ADBM parameterised using rejection ABC plotted against the mean true skill statistic for each food webs. The vertical and horizontal bars correspond to 95\% prediction intervals of the standardised error and true skill statistic respectively. Solid blue line is linear regression through the means (t = -2.44, df = 14, P = 0.028). (b) The mean standardised error computed from the ABC method plotted against the mean standardised error from Petchey et al. (2008). The dashed line is the 1:1 relationship for reference.}\label{fig:unnamed-chunk-8}
\end{figure}

Within each food web, we found various relationships between the
standardised error and true skill statistic (SI Figs S37 and S38). E.g.
For Skipwith Pond food web (SI Fig. S37(l)), high values of TSS were
associated with high error, whereas the opposite was true for other food
webs, such as Broadstone Stream (SI Fig. S37(b, p)). The other food webs
showed more complex relationships.

\hypertarget{discussion}{%
\section{Discussion}\label{discussion}}

The ABC parameterisation method employed here improves on the basic
parameterisation methods applied in Petchey et al. (2008) (PBRW). The
ABC method provides uncertainty in parameter estimates, and thereby a
range of predicted food webs (Fig. \ref{fig:fig_r2}(c-f)). It also
allowed us to estimate parameters that were fixed by the PBRW study, and
thereby also predicts connectance (Fig. \ref{fig:fig_r2}(b)). Including
uncertainty and predicting connectance are significant advances in ADBM.
They allow predictions in changes of food web structure caused by
environmental changes that include uncertainty in the predicted food web
structure and including uncertainty in such predictions is critical
(Petchey et al. 2015; Cressie et al. 2009; Lindegren et al. 2010). A
future development will be to partition the contribution of different
sources of uncertainty such as incomplete sampling and model
deficiencies to make improvements in the model with the aim of reducing
uncertainty. Future research should investigate the functional and
dynamical significance of the uncertainty in the predicted food web
structure. Below we discuss some of the results of our study, and expand
on these opportunities and priorities for future research.

In all cases, the predicted connectance was greater than the observed
connectance (Fig. \ref{fig:fig_r2}(b)). Why did this occur? Firstly, it
is important to recognise that the ADBM (when using only body size as a
trait) can only predict diets that are contiguous with respect to the
size of prey. I.e. it cannot predict that a predator will consume prey
of size 1 and 3, and not prey of size 2. Such patterns can however be
predicted if a trait other than size and which is not perfectly
correlated with size, influences foraging parameters (Petchey et al.
2008). Secondly, it is important to note that the observed diets were
not contiguous when prey are ordered by their size. The estimation
process will result in a greater number of predicted links than observed
given these features, and the model attempts to maximise the coincidence
of predicted and observed link presence and absence (i.e.~the true skill
statistic).

These findings raise the question as to whether the model or the
observed data is incorrect. We expect that the observed data does not
contain some links that would occur in reality. This can be possible due
to low sampling effort causing some links that do occur to be not
observed. In this case, the model may correctly predict a link that was
not yet observed as the data was incorrect. More intensive and more
complete sampling of links in food webs has been recognised as
important, due to the potential that a low sampling effort will
influence the perceived food web structure (Martinez et al. 1999).

We expect there are cases where the model incorrectly predicts a feeding
link despite no possibility that such a link would occur in reality.
This may be the case when a trait other than, or in addition to, prey
size is influential. For example, a particular prey species may have a
defensive trait that means it takes longer to consume it than an
undefended prey of the same size. Incorporating traits other than body
size in the ADBM would allow for discontiguous diets along the size
axis. Furthermore, the ADBM's current form is a biology-only model; it
does not include an observation process, although this could be
included. The model would then be able to predict the absence of a link
due to incomplete observations.

It would be interesting to take a very well sampled food web, and test
if the ABC parameter estimation applied to a subset of the observed
links in a simulated poorly sampled food web predicts the connectance of
a well sampled food web. Such an outcome would indicate the potential to
compensate for under-sampling with an appropriate food web model and
estimation procedure.

The ABC parameterisation resulted in a lower prediction accuracy of
structural features of the food webs (Fig. \ref{fig:fig_r5} (b)) due to
the overestimation of connectance. This was confirmed by principal
component analysis of variation in the food web structural properties
which revealed a first PC axis representing on average 62\% of the
overall variance, and this first axis was highly correlated with
connectance, with an average Spearman correlation of 0.87.

Our parameterisation approach was to maximise the true skill statistic
(the coincidence of predicted and observed link presences, and the
coincidence of predicted and observed link absences). The TSS assigns
equal importance to the collection of presence and absence of observed
links with the weight of an observed single presence or absence link
being dependent on the connectance of the food web. If the connectance
is less than 0.5, the TSS assigns more weight to a presence of link than
to an absence of a link and vice versa.

Because the connectance of the observed food webs is less than 0.5
(Table \ref{fig:tab_1}), the TSS assigned more weight to a single
presence of link than to a single absence of link. This result is as
expected, as the chance that a recorded link is a correct is likely to
be greater than the chance that a recorded absence is correct. This is
because the observation of a single feeding interaction is sufficient to
record the presence of a link. However, this is not true for the absence
of links: one observation of a predator not consuming a prey does not
mean that it never do so. Nevertheless, if we observe no interaction
between two species during the sampling period, we conclude that there
is an absence of link.

To improve our estimation procedure we could quantify the uncertainty in
the recorded absence of links and include this uncertainty in the
parameterisation method. Weight/importance could be assigned to true
positives, true negatives, false positives and false negatives
calculated from empirical studies which may be specific to that food
web. Alternatively, an observation process could be added to the model,
such that the biological part of the model can predict that a feeding
link is possible, but then the observation process in the model leads to
that link not being predicted.

In the PBRW study, the parameter \(b\) played a major role in
maintaining the maximum predictive power of the ADBM. Indeed, they found
that estimating \(b\) only, and not estimating either \(a_i\) or \(a_j\)
slightly decreased model performance, and that estimating only \(b\) and
\(a_j\) did not decrease model performance relative to when all three
parameters were estimated.

We found that the posterior distribution of the parameter \(b\) was the
most constrained of all the parameters (Fig. \ref{fig:fig_r3}).
Parameter \(b\) defines the range of prey body size which has a finite
handling time, and the prey size with the highest energetic
profitability. As the parameter \(b\) relates to the prey-predator body
size ratio, the constrained posterior of \(b\) (Fig.
\ref{fig:fig_r3}(d)) indicates the importance of the ratio of body size
of prey and predator in determining the food web structure with the
ADBM.

The marginal posterior of parameter \(a\) was right-skewed (Fig.
\ref{fig:fig_r3}(c)). This may be because the ABC parameterisation
overestimates the connectance, which means that lower values of \(a\)
are preferred over higher values of \(a\) (a lower value of \(a\) leads
to a lower space clearance/attack rate, and a lower space clearance rate
results in a higher connectance).

Information about who eats who can be collected from multiple sources,
such as gut contents of organisms, stable isotope composition of
tissues, and experimentation (Peralta-Maraver, L\a'opez-Rodríguez, and
de Figueroa 2017; Layman et al. 2007; Warren 1989). Moreover,
experimentation provides independent estimates of allometric foraging
parameters, such as \(b\), \(a_i\), and \(a_j\) (Rall et al. 2012).
Diverse data could be used to parameterise the ADBM's predictions to
test how uncertainty in the different datasets influences the ADBM's
predictions using ABC. Appropriate summary statistics in the ABC method
could be used to address such challenges. We could use, as an example,
the approximate trophic position inferred from stable isotope ratio data
from an individual tissue and gut content data of a predator
simultaneously to parameterise the ADBM. The trophic position and the
gut content information would be the summary statistics in this example.
A further question that could be addressed in future studies is how the
quantity of data affects the ADBM's predictions. The outcome of such a
study could help food web researchers decide on how much data from a
specific source is needed to predict the food web structure, and help
further optimise the deployment of limited sampling resources.

When only partial food web data is available (Patonai and Jord\a'an
2017), the summary statistics in ABC can be used to infer these food web
structures from the ADBM. It would be possible to use gut content data
of only some of the species in a food web to parameterise the ADBM and
predict the food web structure. Summary statistics opens up a broad
spectrum of possibilities in parameterising food web models. There are
multiple empirical and theoretical studies on a range of different food
web properties of food webs across different ecosystems (Williams and
Martinez 2000; Goldwasser and Roughgarden 1993; Martinez 1991). These
can conceivably be used in parameterising food web models using ABC to
constrain the model predictions.

\hypertarget{acknowledgements}{%
\section{Acknowledgements}\label{acknowledgements}}

This work was supported by the University Research Priority Program
Global Change and Biodiversity (Grant number: U-704-04-11) of the
University of Zurich. We thank the Petchey group members for their
valuable suggestions in the manuscript. We thank Debra Zuppinger-Dingley
for proofreading the manuscript.

\hypertarget{author-contributions}{%
\section{Author contributions}\label{author-contributions}}

Anubhav Gupta: Conceptualization (equal), formal analysis (lead),
methodology (lead), software (lead), writing -- original draft
preparation (lead), writing - review and editing (equal). Owen L.
Petchey: Conceptualization (equal), funding acquisition (lead),
methodology (supporting), resources (lead), writing -- original draft
preparation (supporting), writing - review and editing (equal).

\hypertarget{references}{%
\section*{References}\label{references}}
\addcontentsline{toc}{section}{References}

\hypertarget{refs}{}
\leavevmode\hypertarget{ref-allesinaGeneralModelFood2008}{}%
Allesina, Stefano, David Alonso, and Mercedes Pascual. 2008. ``A General
Model for Food Web Structure.'' \emph{Science} 320 (5876). American
Association for the Advancement of Science: 658--61.
\url{https://doi.org/10.1126/science.1156269}.

\leavevmode\hypertarget{ref-bakerFishGutContent2014}{}%
Baker, Ronald, Amanda Buckland, and Marcus Sheaves. 2014. ``Fish Gut
Content Analysis: Robust Measures of Diet Composition.'' \emph{Fish and
Fisheries} 15 (1): 170--77. \url{https://doi.org/10.1111/faf.12026}.

\leavevmode\hypertarget{ref-beckermanForagingBiologyPredicts2006}{}%
Beckerman, A. P., O. L. Petchey, and P. H. Warren. 2006. ``Foraging
Biology Predicts Food Web Complexity.'' \emph{Proceedings of the
National Academy of Sciences of the United States of America} 103:
13745--9.

\leavevmode\hypertarget{ref-bergaminoFoodWebStructure2011}{}%
Bergamino, Leandro, Diego Lercari, and Omar Defeo. 2011. ``Food Web
Structure of Sandy Beaches: Temporal and Spatial Variation Using Stable
Isotope Analysis.'' \emph{Estuarine, Coastal and Shelf Science} 91 (4):
536--43. \url{https://doi.org/10.1016/j.ecss.2010.12.007}.

\leavevmode\hypertarget{ref-broseBodySizesConsumers2005}{}%
Brose, Ulrich, Lara Cushing, Eric L. Berlow, Tomas Jonsson, Carolin
Banasek-Richter, Louis-Felix Bersier, Julia L. Blanchard, et al. 2005.
``Body Sizes of Consumers and Their Resources.'' \emph{Ecology} 86 (9):
2545--5. \url{https://doi.org/10.1890/05-0379}.

\leavevmode\hypertarget{ref-carpenterEcologicalFuturesBuilding2016}{}%
Carpenter, Stephen R. 2016. ``Ecological Futures: Building an Ecology of
the Long Now.'' \emph{Ecology}, October, 2069--83.
\href{https://doi.org/10.1890/0012-9658(2002)083\%5B2069:EFBAEO\%5D2.0.CO;2@10.1002/(ISSN)1939-9170.MacArthurAward}{https://doi.org/10.1890/0012-9658(2002)083{[}2069:EFBAEO{]}2.0.CO;2@10.1002/(ISSN)1939-9170.MacArthurAward}.

\leavevmode\hypertarget{ref-cattinPhylogeneticConstraintsAdaptation2004}{}%
Cattin, Marie-France, Louis-F\a'elix Bersier, Carolin Banašek-Richter,
Richard Baltensperger, and Jean-Pierre Gabriel. 2004. ``Phylogenetic
Constraints and Adaptation Explain Food-Web Structure.'' \emph{Nature}
427 (6977, 6977). Nature Publishing Group: 835--39.
\url{https://doi.org/10.1038/nature02327}.

\leavevmode\hypertarget{ref-cohenJustProportionsFood1989}{}%
Cohen, Joel E. 1989. ``Just Proportions in Food Webs.'' \emph{Nature}
341 (6238, 6238). Nature Publishing Group: 104--5.
\url{https://doi.org/10.1038/341104b0}.

\leavevmode\hypertarget{ref-cohenStochasticTheoryCommunity1985}{}%
Cohen, Joel E., C. M. Newman, and John Hyslop Steele. 1985. ``A
Stochastic Theory of Community Food Webs I. Models and Aggregated
Data.'' \emph{Proceedings of the Royal Society of London. Series B.
Biological Sciences} 224 (1237). Royal Society: 421--48.
\url{https://doi.org/10.1098/rspb.1985.0042}.

\leavevmode\hypertarget{ref-crawfordApplicationsStableIsotope2008}{}%
Crawford, Kerry, Robbie A. Mcdonald, and Stuart Bearhop. 2008.
``Applications of Stable Isotope Techniques to the Ecology of Mammals.''
\emph{Mammal Review} 38 (1): 87--107.
\url{https://doi.org/10.1111/j.1365-2907.2008.00120.x}.

\leavevmode\hypertarget{ref-cressieAccountingUncertaintyEcological2009}{}%
Cressie, Noel, Catherine A. Calder, James S. Clark, Jay M. Ver Hoef, and
Christopher K. Wikle. 2009. ``Accounting for Uncertainty in Ecological
Analysis: The Strengths and Limitations of Hierarchical Statistical
Modeling.'' \emph{Ecological Applications} 19 (3): 553--70.
\url{https://doi.org/10.1890/07-0744.1}.

\leavevmode\hypertarget{ref-dawahStructureParasitoidCommunities1995}{}%
Dawah, Hassan Ali, Bradford A. Hawkins, and Michael F. Claridge. 1995.
``Structure of the Parasitoid Communities of Grass-Feeding Chalcid
Wasps.'' \emph{The Journal of Animal Ecology} 64 (6): 708.
\url{https://doi.org/10.2307/5850}.

\leavevmode\hypertarget{ref-dunneNetworkStructureBiodiversity}{}%
Dunne, Jennifer A., Richard J. Williams, and Neo D. Martinez. 2002.
``Network Structure and Biodiversity Loss in Food Webs: Robustness
Increases with Connectance.'' \emph{Ecology Letters} 5 (4): 558--67.

\leavevmode\hypertarget{ref-emmersonPredatorPreyBody2004}{}%
Emmerson, Mark C., and Dave Raffaelli. 2004. ``Predator--Prey Body Size,
Interaction Strength and the Stability of a Real Food Web.''
\emph{Journal of Animal Ecology} 73 (3): 399--409.
\url{https://doi.org/10.1111/j.0021-8790.2004.00818.x}.

\leavevmode\hypertarget{ref-goldwasserConstructionAnalysisLarge1993}{}%
Goldwasser, Lloyd, and Jonathan Roughgarden. 1993. ``Construction and
Analysis of a Large Caribbean Food Web: Ecological Archives E074-001.''
\emph{Ecology} 74 (4): 1216--33. \url{https://doi.org/10.2307/1940492}.

\leavevmode\hypertarget{ref-gravelInferringFoodWeb2013}{}%
Gravel, Dominique, Timoth\a'ee Poisot, Camille Albouy, Laure Velez, and
David Mouillot. 2013. ``Inferring Food Web Structure from Predator-Prey
Body Size Relationships.'' Edited by Robert Freckleton. \emph{Methods in
Ecology and Evolution} 4 (11): 1083--90.
\url{https://doi.org/10.1111/2041-210X.12103}.

\leavevmode\hypertarget{ref-harper-smithCOMMUNICATINGECOLOGYFOOD2005}{}%
Harper-Smith, Sarah, Eric L. Berlow, Roland A. Knapp, Richard J.
Williams, and Neo D. Martinez. 2005. ``COMMUNICATING ECOLOGY THROUGH
FOOD WEBS: VISUALIZING AND QUANTIFYING THE EFFECTS OF STOCKING ALPINE
LAKES WITH TROUT.'' In \emph{Dynamic Food Webs}, 407--23. Elsevier.
\url{https://doi.org/10.1016/B978-012088458-2/50038-2}.

\leavevmode\hypertarget{ref-hattabForecastingFinescaleChanges2016}{}%
Hattab, Tarek, Fabien Leprieur, Frida Ben Rais Lasram, Dominique Gravel,
François Le Loc'h, and Camille Albouy. 2016. ``Forecasting Fine-Scale
Changes in the Food-Web Structure of Coastal Marine Communities Under
Climate Change.'' \emph{Ecography} 39 (12): 1227--37.
\url{https://doi.org/10.1111/ecog.01937}.

\leavevmode\hypertarget{ref-hobsonUsingStableIsotopes1994}{}%
Hobson, Keith A., John F. Piatt, and Jay Pitocchelli. 1994. ``Using
Stable Isotopes to Determine Seabird Trophic Relationships.''
\emph{Journal of Animal Ecology} 63 (4). {[}Wiley, British Ecological
Society{]}: 786--98. \url{https://doi.org/10.2307/5256}.

\leavevmode\hypertarget{ref-hudsonCheddarAnalysisVisualisation2013}{}%
Hudson, Lawrence N., Rob Emerson, Gareth B. Jenkins, Katrin Layer, Mark
E. Ledger, Doris E. Pichler, Murray S. A. Thompson, Eoin J. O'Gorman,
Guy Woodward, and Daniel C. Reuman. 2013. ``Cheddar: Analysis and
Visualisation of Ecological Communities in R.'' \emph{Methods in Ecology
and Evolution} 4 (1): 99--104.
\url{https://doi.org/10.1111/2041-210X.12005}.

\leavevmode\hypertarget{ref-ibanezOptimizingSizeThresholds2012}{}%
Ibanez, S\a'ebastien. 2012. ``Optimizing Size Thresholds in a
Plant--Pollinator Interaction Web: Towards a Mechanistic Understanding
of Ecological Networks.'' \emph{Oecologia} 170 (1): 233--42.
\url{https://doi.org/10.1007/s00442-012-2290-3}.

\leavevmode\hypertarget{ref-jabotInferringParametersNeutral2009}{}%
Jabot, Franck, and J\a'erôme Chave. 2009. ``Inferring the Parameters of
the Neutral Theory of Biodiversity Using Phylogenetic Information and
Implications for Tropical Forests.'' \emph{Ecology Letters} 12 (3):
239--48. \url{https://doi.org/10.1111/j.1461-0248.2008.01280.x}.

\leavevmode\hypertarget{ref-JonssonPhD}{}%
Jonsson, Tomas. 1998. ``Food Webs and the Distribution of Body Sizes.''
PhD.

\leavevmode\hypertarget{ref-jonssonFoodWebsBody2005}{}%
Jonsson, Tomas, Joel E. Cohen, and Stephen R. Carpenter. 2005. ``Food
Webs, Body Size, and Species Abundance in Ecological Community
Description.'' In \emph{Advances in Ecological Research}, 36:1--84.
Elsevier. \url{https://doi.org/10.1016/S0065-2504(05)36001-6}.

\leavevmode\hypertarget{ref-jordanSensitivityFoodWeb2009}{}%
Jord\a'an, Ferenc, and Györgyi Osv\a'ath. 2009. ``The Sensitivity of
Food Web Topology to Temporal Data Aggregation.'' \emph{Ecological
Modelling} 220 (22): 3141--6.
\url{https://doi.org/10.1016/j.ecolmodel.2009.05.002}.

\leavevmode\hypertarget{ref-knightTrophicCascadesEcosystems2005}{}%
Knight, Tiffany M., Michael W. McCoy, Jonathan M. Chase, Krista A.
McCoy, and Robert D. Holt. 2005. ``Trophic Cascades Across Ecosystems.''
\emph{Nature} 437 (7060): 880--83.
\url{https://doi.org/10.1038/nature03962}.

\leavevmode\hypertarget{ref-krauseCompartmentsRevealedFoodweb2003}{}%
Krause, Ann E., Kenneth A. Frank, Doran M. Mason, Robert E. Ulanowicz,
and William W. Taylor. 2003. ``Compartments Revealed in Food-Web
Structure.'' \emph{Nature} 426 (6964): 282--85.
\url{https://doi.org/10.1038/nature02115}.

\leavevmode\hypertarget{ref-laffertyParasitesDominateFood2006}{}%
Lafferty, K. D., A. P. Dobson, and A. M. Kuris. 2006. ``Parasites
Dominate Food Web Links.'' \emph{Proceedings of the National Academy of
Sciences} 103 (30): 11211--6.
\url{https://doi.org/10.1073/pnas.0604755103}.

\leavevmode\hypertarget{ref-laymanCanStableIsotope2007}{}%
Layman, Craig A., D. Albrey Arrington, Carmen G. Montaña, and David M.
Post. 2007. ``Can Stable Isotope Ratios Provide for Community-Wide
Measures of Trophic Structure?'' \emph{Ecology} 88 (1): 42--48.
\href{https://doi.org/10.1890/0012-9658(2007)88\%5B42:CSIRPF\%5D2.0.CO;2}{https://doi.org/10.1890/0012-9658(2007)88{[}42:CSIRPF{]}2.0.CO;2}.

\leavevmode\hypertarget{ref-lindegrenEcologicalForecastingClimate2010}{}%
Lindegren, Martin, Christian Möllmann, Anders Nielsen, Keith Brander,
Brian R. MacKenzie, and Nils Chr. Stenseth. 2010. ``Ecological
Forecasting Under Climate Change: The Case of Baltic Cod.''
\emph{Proceedings of the Royal Society B: Biological Sciences} 277
(1691): 2121--30. \url{https://doi.org/10.1098/rspb.2010.0353}.

\leavevmode\hypertarget{ref-lurgiClimateChangeImpacts2012}{}%
Lurgi, Miguel, Bernat C. L\a'opez, and Jos\a'e M. Montoya. 2012.
``Climate Change Impacts on Body Size and Food Web Structure on Mountain
Ecosystems.'' \emph{Philosophical Transactions of the Royal Society B:
Biological Sciences} 367 (1605): 3050--7.
\url{https://doi.org/10.1098/rstb.2012.0239}.

\leavevmode\hypertarget{ref-macarthurOptimalUsePatchy1966}{}%
MacArthur, Robert H., and Eric R. Pianka. 1966. ``On Optimal Use of a
Patchy Environment.'' \emph{The American Naturalist} 100 (916): 603--9.

\leavevmode\hypertarget{ref-martinezArtifactsAttributesEffects1991}{}%
Martinez, Neo D. 1991. ``Artifacts or Attributes? Effects of Resolution
on the Little Rock Lake Food Web.'' \emph{Ecological Monographs} 61 (4):
367--92. \url{https://doi.org/10.2307/2937047}.

\leavevmode\hypertarget{ref-martinezEffectsSamplingEffort1999}{}%
Martinez, Neo D., Bradford A. Hawkins, Hassan Ali Dawah, and Brian P.
Feifarek. 1999. ``Effects of Sampling Effort on Characterization of
Food-Web Structure.'' \emph{Ecology} 80 (3): 1044--55.
\href{https://doi.org/10.1890/0012-9658(1999)080\%5B1044:EOSEOC\%5D2.0.CO;2}{https://doi.org/10.1890/0012-9658(1999)080{[}1044:EOSEOC{]}2.0.CO;2}.

\leavevmode\hypertarget{ref-mayWillLargeComplex1972}{}%
May, Robert M. 1972. ``Will a Large Complex System Be Stable?''
\emph{Nature} 238 (5364): 413. \url{https://doi.org/10.1038/238413a0}.

\leavevmode\hypertarget{ref-memmottPredatorsParasitoidsPathogens2000}{}%
Memmott, J., N.D. Martinez, and J.E. Cohen. 2000. ``Predators,
Parasitoids and Pathogens: Species Richness, Trophic Generality and Body
Sizes in a Natural Food Web.'' \emph{Journal of Animal Ecology} 69 (1):
1--15. \url{https://doi.org/10.1046/j.1365-2656.2000.00367.x}.

\leavevmode\hypertarget{ref-morrisFoodWebStructure2015}{}%
Morris, Rebecca J., Frazer H. Sinclair, and Chris J. Burwell. 2015.
``Food Web Structure Changes with Elevation but Not Rainforest
Stratum.'' \emph{Ecography} 38 (8): 792--802.
\url{https://doi.org/10.1111/ecog.01078}.

\leavevmode\hypertarget{ref-oconnorWarmingResourceAvailability2009}{}%
O'Connor, Mary I., Michael F. Piehler, Dina M. Leech, Andrea Anton, and
John F. Bruno. 2009. ``Warming and Resource Availability Shift Food Web
Structure and Metabolism.'' Edited by Michel Loreau. \emph{PLoS Biology}
7 (8): e1000178. \url{https://doi.org/10.1371/journal.pbio.1000178}.

\leavevmode\hypertarget{ref-ogormanSimpleModelPredicts2019}{}%
O'Gorman, Eoin J., Owen L. Petchey, Katy J. Faulkner, Bruno Gallo,
Timothy A. C. Gordon, Joana Neto-Cerejeira, J\a'on S. \a'Olafsson, Doris
E. Pichler, Murray S. A. Thompson, and Guy Woodward. 2019. ``A Simple
Model Predicts How Warming Simplifies Wild Food Webs.'' \emph{Nature
Climate Change} 9 (8): 611--16.
\url{https://doi.org/10.1038/s41558-019-0513-x}.

\leavevmode\hypertarget{ref-opitz1996}{}%
Opitz, Silvia. 1996. ``Quantitative Models of Trophic Interactions in
Caribbean Coral Reefs.'' Iclarm.

\leavevmode\hypertarget{ref-patonaiAggregationIncompleteFood2017}{}%
Patonai, Katalin, and Ferenc Jord\a'an. 2017. ``Aggregation of
Incomplete Food Web Data May Help to Suggest Sampling Strategies.''
\emph{Ecological Modelling} 352 (May): 77--89.
\url{https://doi.org/10.1016/j.ecolmodel.2017.02.024}.

\leavevmode\hypertarget{ref-peralta-maraverStructureDynamicsStability2017}{}%
Peralta-Maraver, I., M. J. L\a'opez-Rodríguez, and J. M. Tierno de
Figueroa. 2017. ``Structure, Dynamics and Stability of a Mediterranean
River Food Web.'' \emph{Marine and Freshwater Research} 68 (3). CSIRO
PUBLISHING: 484--95. \url{https://doi.org/10.1071/MF15154}.

\leavevmode\hypertarget{ref-petchey2008size}{}%
Petchey, Owen L, Andrew P Beckerman, Jens O Riede, and Philip H Warren.
2008. ``Size, Foraging, and Food Web Structure.'' \emph{Proceedings of
the National Academy of Sciences} 105 (11). National Acad Sciences:
4191--6.

\leavevmode\hypertarget{ref-petcheyEnvironmentalWarmingAlters1999}{}%
Petchey, Owen L., P. Timon McPhearson, Timothy M. Casey, and Peter J.
Morin. 1999. ``Environmental Warming Alters Food-Web Structure and
Ecosystem Function.'' \emph{Nature} 402 (6757): 69--72.
\url{https://doi.org/10.1038/47023}.

\leavevmode\hypertarget{ref-petcheyEcologicalForecastHorizon2015}{}%
Petchey, Owen L., Mikael Pontarp, Thomas M. Massie, Sonia K\a'efi, Arpat
Ozgul, Maja Weilenmann, Gian Marco Palamara, et al. 2015. ``The
Ecological Forecast Horizon, and Examples of Its Uses and
Determinants.'' \emph{Ecology Letters} 18 (7): 597--611.
\url{https://doi.org/10.1111/ele.12443}.

\leavevmode\hypertarget{ref-poisotWhenEcologicalNetwork2014}{}%
Poisot, Timoth\a'ee, and Dominique Gravel. 2014. ``When Is an Ecological
Network Complex? Connectance Drives Degree Distribution and Emerging
Network Properties.'' \emph{PeerJ} 2 (February). PeerJ Inc.: e251.
\url{https://doi.org/10.7717/peerj.251}.

\leavevmode\hypertarget{ref-poisotHowEcologicalNetworks2016}{}%
Poisot, Timoth\a'ee, and Daniel B. Stouffer. 2016. ``How Ecological
Networks Evolve.'' Preprint. Ecology.
\url{https://doi.org/10.1101/071993}.

\leavevmode\hypertarget{ref-polisComplexTrophicInteractions1991}{}%
Polis, Gary A. 1991. ``Complex Trophic Interactions in Deserts: An
Empirical Critique of Food-Web Theory.'' \emph{The American Naturalist}
138 (1): 123--55.

\leavevmode\hypertarget{ref-rallUniversalTemperatureBodymass2012}{}%
Rall, B. C., U. Brose, M. Hartvig, G. Kalinkat, F. Schwarzmuller, O.
Vucic-Pestic, and O. L. Petchey. 2012. ``Universal Temperature and
Body-Mass Scaling of Feeding Rates.'' \emph{Philosophical Transactions
of the Royal Society B: Biological Sciences} 367 (1605): 2923--34.
\url{https://doi.org/10.1098/rstb.2012.0242}.

\leavevmode\hypertarget{ref-shrinerEvolutionIntrahostHiv2006}{}%
Shriner, Daniel, Yi Liu, David C. Nickle, and James I. Mullins. 2006.
``Evolution of Intrahost Hiv - 1 Genetic Diversity During Chronic
Infection.'' \emph{Evolution} 60 (6): 1165--76.
\url{https://doi.org/10.1111/j.0014-3820.2006.tb01195.x}.

\leavevmode\hypertarget{ref-tamaddoni-nezhadConstructionValidationFood2013}{}%
Tamaddoni-Nezhad, Alireza, Ghazal Afroozi Milani, Alan Raybould, Stephen
Muggleton, and David A. Bohan. 2013. ``Construction and Validation of
Food Webs Using Logic-Based Machine Learning and Text Mining.'' In
\emph{Advances in Ecological Research}, 49:225--89. Elsevier.
\url{https://doi.org/10.1016/B978-0-12-420002-9.00004-4}.

\leavevmode\hypertarget{ref-toniApproximateBayesianComputation2009}{}%
Toni, Tina, David Welch, Natalja Strelkowa, Andreas Ipsen, and Michael
P.H. Stumpf. 2009. ``Approximate Bayesian Computation Scheme for
Parameter Inference and Model Selection in Dynamical Systems.''
\emph{Journal of the Royal Society Interface} 6 (31): 187--202.
\url{https://doi.org/10.1098/rsif.2008.0172}.

\leavevmode\hypertarget{ref-tylianakisEffectsGlobalEnvironmental2014}{}%
Tylianakis, Jason M., and Amrei Binzer. 2014. ``Effects of Global
Environmental Changes on ParasitoidHost Food Webs and Biological
Control.'' \emph{Biological Control} 75 (August): 77--86.
\url{https://doi.org/10.1016/j.biocontrol.2013.10.003}.

\leavevmode\hypertarget{ref-warrenSpatialTemporalVariation1989}{}%
Warren, Philip H. 1989. ``Spatial and Temporal Variation in the
Structure of a Freshwater Food Web.'' \emph{Oikos} 55 (3): 299.
\url{https://doi.org/10.2307/3565588}.

\leavevmode\hypertarget{ref-williamsSimpleRulesYield2000}{}%
Williams, Richard J., and Neo D. Martinez. 2000. ``Simple Rules Yield
Complex Food Webs.'' \emph{Nature} 404 (6774, 6774). Nature Publishing
Group: 180--83. \url{https://doi.org/10.1038/35004572}.

\leavevmode\hypertarget{ref-woodwardChapterIndividualBasedFood2010}{}%
Woodward, Guy, Julia Blanchard, Rasmus B. Lauridsen, Francois K.
Edwards, J. Iwan Jones, David Figueroa, Philip H. Warren, and Owen L.
Petchey. 2010. ``Chapter 6 - Individual-Based Food Webs: Species
Identity, Body Size and Sampling Effects.'' In \emph{Advances in
Ecological Research}, edited by Guy Woodward, 43:211--66. Integrative
Ecology: From Molecules to Ecosystems. Academic Press.
\url{https://doi.org/10.1016/B978-0-12-385005-8.00006-X}.

\leavevmode\hypertarget{ref-woodwardInvasionStreamFood2001}{}%
Woodward, Guy, and Alan G. Hildrew. 2001. ``Invasion of a Stream Food
Web by a New Top Predator.'' \emph{Journal of Animal Ecology} 70 (2):
273--88. \url{https://doi.org/10.1111/j.1365-2656.2001.00497.x}.

\leavevmode\hypertarget{ref-woodwardQuantificationResolutionComplex2005}{}%
Woodward, Guy, Dougie C. Speirs, and Alan G. Hildrew. 2005.
``Quantification and Resolution of a Complex, Size-Structured Food
Web.'' In \emph{Advances in Ecological Research}, 36:85--135. Elsevier.
\url{https://doi.org/10.1016/S0065-2504(05)36002-8}.

\leavevmode\hypertarget{ref-yodzisLocalTrophodynamicsInteraction1998}{}%
Yodzis, Peter. 1998. ``Local Trophodynamics and the Interaction of
Marine Mammals and Fisheries in the Benguela Ecosystem.'' \emph{Journal
of Animal Ecology} 67 (4): 635--58.
\url{https://doi.org/10.1046/j.1365-2656.1998.00224.x}.

\bibliographystyle{unsrt}
\bibliography{bibliography.bib}


\end{document}
